\documentclass[12pt,a4paper]{article}

\usepackage{geometry}
\geometry{
    left=2cm, 
    right=2cm,
    top=3cm,  
    bottom=2cm
}

\usepackage[english]{babel}
\usepackage[utf8]{inputenc}
\usepackage{amsmath}

\usepackage{graphicx}

\usepackage{setspace}
\setstretch{1.25}
\setlength{\parindent}{0pt}

\usepackage{enumitem}


\begin{document}
    
    \begin{titlepage}
        \vspace*{\fill}
        \begin{center}
            {\Huge Usage Manual of\\Recommender System}
        \end{center}
        \vspace*{\fill}
    \end{titlepage}

    \newpage

    \tableofcontents

    \newpage

    \section{Generalities}
    {
        In this section are found different procedures and basic concepts that 
        are shared throughout the different profiles and stages.

        \subsection{Initial Installation}
        {
            To start developing a feature, test a functionality or module, or 
            deploy the service, it is required to clone the repository, for 
            this purpose it is used:
            
            \begin{verbatim}
            git clone https://github.com/alexisuaguilaru/BookExplorer.git 
            \end{verbatim}

            Each of the modules that are written in Python, have a 
            \verb*|requirements.txt| file that contains the libraries required 
            for the operation of the module or, alternatively, to install to 
            develop or implement a change in a feature of the respective module.
        }

        \subsection{Docker Compose}
        {
            The service is divided into three files \verb*|docker-compose.yml|, 
            where each one contains the different microservices according to the 
            functionality it performs within the service. The files contain the 
            following microservices:

            \begin{itemize}[label=$\bullet$]
                \item \textbf{Backend}: MongoDB database [BooksDatabase], Data extraction 
                from the API [DataExtraction] and API of the main microservice [SystemRecommender]
                \item \textbf{Frontend}: Service web interface [WebInterface] 
                \item \textbf{Proxy}: Reverse proxy based on nginx [Proxy] and automation of SSL 
                certificate acquisition [Certbot]
            \end{itemize}
        }

        \subsection{Configuration File \emph{.env}}
        {
            The \verb*|.evn_example| file serves as an example of the environment 
            variables that must be changed to adjust the behavior of the three 
            \verb*|docker-compose.yml| files. The configurable environment variables are:

            \begin{itemize}[label=$\bullet$]
                \item \emph{UID}: ID of the non root user that executes the different services in the \verb*|docker-compose|
                \item \emph{GID}: ID of the group the non root user belongs to
                \item \emph{MONGO\_INITDB\_ROOT\_USERNAME}: Username of the database administrator in MongoDB
                \item \emph{MONGO\_INITDB\_ROOT\_PASSWORD}: Password for the MongoDB administrator
                \item \emph{DB\_NAME}: Name of the database to be created in MongoDB
                \item \emph{MONGO\_WRITE\_USERNAME}: Username of non root user who can only write data to the database
                \item \emph{MONGO\_WRITE\_PASSWORD}: Password for user with write role
                \item \emph{MONGO\_READ\_USERNAME}: Username of non root user who only reads data in the database
                \item \emph{MONGO\_READ\_PASSWORD}: Password for the user with read role
                \item \emph{AMOUNT\_BOOKS}: Number of books that are extracted from the API. It does not represent the final number found in the associated collection in MongoDB.
                \item \emph{USER\_AGENT}: Identification credentials for the OpenLibrary API (project name, email). It is not a token or key for the API.
                \item \emph{FLASK\_SECRET\_KEY}: Secret key used by Flask for encryption and session management.
                \item \emph{DEBUG\_MODE}: Mode in which Flask applications run, whether in debug mode or production mode
                \item \emph{DOMAIN\_NAME}: Name of the domain where the service will be served
                \item \emph{EMAIL\_ID}: Email used to generate certificates generated from CertBot
            \end{itemize}
        }

        \subsection{Execution File \emph{Makefile}}
        {
            The \verb*|Makefile| file contains commands that execute the different 
            services contained in the \verb*|docker-compose.yml| files depending on 
            whether it is the first time it is executed or not. It contains the 
            following variables to configure:

            \begin{itemize}[label=$\bullet$]
                \item \emph{DOMAIN\_NAME}: Name of the domain where the service will be served and from where the self-signed certificates are generated.
                \item \emph{ENV}: Environment under which the service is being executed, depending on whether it is for development, testing or deployment.
            \end{itemize}

            Since it deals with both passwords and secret keys, the following command 
            is available to generate them using alphanumeric characters:

            \begin{verbatim}
            make secret_key LENGTH=integer
            \end{verbatim}

            Where \emph{LENGTH} controls the length of the key, preferably greater 
            than $12$ characters.
        }

    }

    \newpage

    \section{Development Stage}

    \newpage

    \section{Testing Stage}

    \newpage

    \section{Deployment Stage}

\end{document}